% Beamer presentation for SSP-NPM GA results
\documentclass{beamer}
\mode<presentation>{
  \usetheme{Madrid}
}
\usepackage[utf8]{inputenc}
\usepackage{graphicx}
\usepackage{booktabs}
\usepackage{listings}
\usepackage{color}
\usepackage{hyperref}

\definecolor{codegray}{rgb}{0.95,0.95,0.95}
\lstset{
  backgroundcolor=\color{codegray},
  basicstyle=\ttfamily\footnotesize,
  breaklines=true,
  columns=fullflexible,
  frame=single,
  framerule=0pt
}

\title[AG Monoobjetivo - SSP-NPM]{Algoritmo Genético Monoobjetivo para SSP-NPM}
\author{Igor Falco}
\institute{Projeto: SSP-NPM / Implementação DEAP}
\date{\today}

\begin{document}

\begin{frame}
  \titlepage
\end{frame}

\begin{frame}{Sumário}
  \tableofcontents
\end{frame}

\section{Introdução}
\begin{frame}{Introdução}
  \begin{itemize}
    \item Problema: SSP-NPM (Sequenciamento e Atribuição de Jobs a Máquinas com restrições de ferramentas e troca de ferramentas).
    \item Objetivo desta versão: minimizar o \textbf{makespan} (problema monoobjetivo).
    \item Instâncias: coleções em `instances/SSP-NPM-I` e `instances/SSP-NPM-II`.
  \end{itemize}
\end{frame}

\section{Motivação}
\begin{frame}{Motivação}
  \begin{itemize}
    \item IC implementando VNS com busca local para soluções multiobjetivas.
    \item Objetivo: criar uma código monoobjetivo utilizando computação evolucionária e comparar os resultados.
    \item Vantagens: Conseguir rodar todas as instãncias em um dia.
  \end{itemize}
\end{frame}

\section{Algoritmo}
\begin{frame}[fragile]{Algoritmo Utilizado}
  \footnotesize
  \begin{itemize}
    \item \textbf{Biblioteca:} \texttt{DEAP} (Algoritmos Evolucionários).
    \item \textbf{Representação:} matriz máquina × posição, usando \texttt{0} para posições vazias.
      \begin{center}
    \includegraphics[width=0.5\textwidth]{tabela.png}
  \end{center}
    \item \textbf{Operadores:}
      \begin{itemize}
        \item Seleção por torneio
        \item Crossover entre máquinas
        \item Mutação por swap dentro da máquina
      \end{itemize}
    \item \textbf{Fitness:} cálculo do makespan usando rotinas otimizadas com Numba.
  \end{itemize}

\end{frame}


\section{Exemplos}
\begin{frame}{Exemplo (Instância pequena)}
  \begin{itemize}
    \item Instância exemplo: `ins1-m=2-j=10-t=10-var=1` (SSP-NPM-I)
    \item Evolução do fitness durante as gerações:
  \end{itemize}
  \begin{center}
    \includegraphics[width=0.9\textwidth]{ins1_m=2_j=10_t=10_var=1.png}
  \end{center}
\end{frame}

\begin{frame}{Exemplo (Instância média)}
  \begin{itemize}
    \item Instância exemplo: `ins161-m=4-j=40-t=60-sw=l-dens=s-var=1` (SSP-NPM-II)
    \item Evolução do fitness durante as gerações:
  \end{itemize}
  \begin{center}
    \includegraphics[width=0.9\textwidth]{ins161_m=4_j=40_t=60_sw=l_dens=s_var=1.png}
  \end{center}
\end{frame}

\begin{frame}{Exemplo (Instância grande)}
  \begin{itemize}
    \item Instância exemplo: `ins640-m=6-j=120-t=120-sw=h-dens=d-var=20` (SSP-NPM-II)
    \item Evolução do fitness durante as gerações:
  \end{itemize}
  \begin{center}
    \includegraphics[width=0.9\textwidth]{ins640_m=6_j=120_t=120_sw=h_dens=d_var=20.png}
  \end{center}
\end{frame}

\section{Resultados e Observações}
\begin{frame}{Resultados e Observações}
  \begin{itemize}
    \item Observação: Não foi possível comparar com o código VNS original porque ele ainda não estava pronto.
    \item Tempo: execução completa (ambos conjuntos) projetada para poucas horas; execução parcial de `SSP-NPM-I` (160 instâncias) levou minutos em máquina local.
  \end{itemize}
\end{frame}

\begin{frame}{Comparação Rápida com VNS}
  \begin{itemize}
    \item Usei os resultados VNS disponíveis em `vns/` e coletei o menor makespan por instância.
  \end{itemize}
  \vspace{2mm}
  \begin{tabular}{l c c c c}
    	oprule
    Instância & GA (best) & VNS (best) & Diferença & Melhor \\
    \midrule
    ins1_m=2_j=10_t=10_var=1 & 34 & 32 & 2 & VNS \\
    ins121_m=3_j=15_t=20_var=1 & 45 & 41 & 4 & VNS \\
    \bottomrule
  \end{tabular}
  \vspace{3mm}
  \small Observação: o valor VNS é o menor makespan observado entre todos os `pareto_wall_run_*.csv` na pasta `vns/`.
\end{frame}

\section{Como Reproduzir}
\begin{frame}{Como Reproduzir}
  \begin{itemize}
    \item Instalar dependências: \texttt{pip install -r src/requirements.txt}
    \item Executar: entrar em `src/` e rodar \texttt{python main_deap.py}
    \item Saída: resultados salvos em `src/results/` por pasta e instância.
  \end{itemize}
\end{frame}

\section{Conclusão}
\begin{frame}{Conclusão}
  \begin{itemize}
    \item Implementamos rapidamente o algoritmo monoobjetivo usando DEAP e as rotinas Numba para avaliação.
    \item O algoritmo converge muito mais rápido que os algoritmos tradicionais.
    \item Próximo passo: Implementar o VNS como forma de refinamento para a solução encontrada pelo algoritmo genético.
  \end{itemize}
\end{frame}

\begin{frame}{Perguntas}
  \begin{center}
    \LARGE Obrigado! \\[6pt]
  \end{center}
\end{frame}

\end{document}
